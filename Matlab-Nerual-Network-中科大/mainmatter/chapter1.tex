\chapter{前向神经网络}

\section{多层感知机}

\begin{question}
    设计感知器实现26个英文字母识别网络,实现训练和测试的步骤。训练包括无噪声的理想字母识别网络和含有不同的噪声的识别网络,使得网络具有“理想+噪声”的抗噪能力。在测试步骤中,分别测试在随机加噪声逐步增加的情况下,测试数据为100组的网络的平均误差,与
    利息那个样本值进行比较,得到不同噪声水平下的误差值,作图进行性能对比。
\end{question}

\section{BP神经网络设计}

\begin{question}
    根据$BP$神经网络设计实现模拟控制规则为$T=int\left(\frac{e+ec}{2}\right)$的模糊神经网络控制器,其中输入变量$e$和输出变量$ec$的变化范围分别是
    \begin{equation}
        \begin{aligned}
            & e\sim \{-2,2\}\\
            & ec\sim \{-2,2\}
        \end{aligned}
    \end{equation}

    网络设计目标误差$e_{min}=0.001$,试着给出
    \begin{enumerate}[itemindent=2em]
        \item 输入输出矢量以及问题描述;
        \item 网络结构;
        \item 学习方法;
        \item 初始化以及必要的参数选取;
        \item 最后的结果、循环次数、训练时间,其中着重讨论:
        \begin{enumerate}[itemindent=2em]
            \item 不同隐藏层$S1$时的收敛速度和误差精度的对比分析;
            \item 在$S1$设置为较好的情况下,对固定学习速率$lr$取不同值时的训练时间,其中包括稳定性进行观察比较;
            \item 采用自适应学习速率,于单一固定的学习速率$lr$中最好的情况进行比较;
        \end{enumerate}
        \item 采用插值法选取多余训练时的输入,对所设计网络进行验证,给出验证的$A$与$T$的值。
    \end{enumerate}
\end{question}

\section{异或问题}

\begin{question}
    采用单层感知机的权值训练公式,通过固定隐藏层的目标输出为线性可分矢量,设计一个两层感知机来解决异或问题,隐藏层用两个神经元。
\end{question}

\vspace*{2em}

\section{线性网络}

\begin{question}
    设计一个具有单元输入和单元输出的线性网络,注意观察其解的特性
    \begin{equation}
        P=1,\ \ \ \ T=0.5
    \end{equation}
\end{question}

\section{附加动量法和自适应学习速率技术}

\begin{question}
    用函数\textsc{trainbpx.m}训练例2.10,并与其他训练方法做比较;
\end{question}